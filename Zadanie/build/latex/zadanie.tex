%% Generated by Sphinx.
\def\sphinxdocclass{report}
\documentclass[letterpaper,10pt,polish]{sphinxmanual}
\ifdefined\pdfpxdimen
   \let\sphinxpxdimen\pdfpxdimen\else\newdimen\sphinxpxdimen
\fi \sphinxpxdimen=.75bp\relax
\ifdefined\pdfimageresolution
    \pdfimageresolution= \numexpr \dimexpr1in\relax/\sphinxpxdimen\relax
\fi
%% let collapsible pdf bookmarks panel have high depth per default
\PassOptionsToPackage{bookmarksdepth=5}{hyperref}

\PassOptionsToPackage{booktabs}{sphinx}
\PassOptionsToPackage{colorrows}{sphinx}

\PassOptionsToPackage{warn}{textcomp}
\usepackage[utf8]{inputenc}
\ifdefined\DeclareUnicodeCharacter
% support both utf8 and utf8x syntaxes
  \ifdefined\DeclareUnicodeCharacterAsOptional
    \def\sphinxDUC#1{\DeclareUnicodeCharacter{"#1}}
  \else
    \let\sphinxDUC\DeclareUnicodeCharacter
  \fi
  \sphinxDUC{00A0}{\nobreakspace}
  \sphinxDUC{2500}{\sphinxunichar{2500}}
  \sphinxDUC{2502}{\sphinxunichar{2502}}
  \sphinxDUC{2514}{\sphinxunichar{2514}}
  \sphinxDUC{251C}{\sphinxunichar{251C}}
  \sphinxDUC{2572}{\textbackslash}
\fi
\usepackage{cmap}
\usepackage[T1]{fontenc}
\usepackage{amsmath,amssymb,amstext}
\usepackage{babel}



\usepackage{tgtermes}
\usepackage{tgheros}
\renewcommand{\ttdefault}{txtt}



\usepackage[Sonny]{fncychap}
\ChNameVar{\Large\normalfont\sffamily}
\ChTitleVar{\Large\normalfont\sffamily}
\usepackage{sphinx}

\fvset{fontsize=auto}
\usepackage{geometry}


% Include hyperref last.
\usepackage{hyperref}
% Fix anchor placement for figures with captions.
\usepackage{hypcap}% it must be loaded after hyperref.
% Set up styles of URL: it should be placed after hyperref.
\urlstyle{same}

\addto\captionspolish{\renewcommand{\contentsname}{Zawartosc:}}

\usepackage{sphinxmessages}
\setcounter{tocdepth}{1}



\title{Zadanie}
\date{02 gru 2025}
\release{1.0}
\author{sobo}
\newcommand{\sphinxlogo}{\vbox{}}
\renewcommand{\releasename}{Wydanie}
\makeindex
\begin{document}

\ifdefined\shorthandoff
  \ifnum\catcode`\=\string=\active\shorthandoff{=}\fi
  \ifnum\catcode`\"=\active\shorthandoff{"}\fi
\fi

\pagestyle{empty}
\sphinxmaketitle
\pagestyle{plain}
\sphinxtableofcontents
\pagestyle{normal}
\phantomsection\label{\detokenize{index::doc}}


\sphinxAtStartPar
Add your content using \sphinxcode{\sphinxupquote{reStructuredText}} syntax. See the
\sphinxhref{https://www.sphinx-doc.org/en/master/usage/restructuredtext/index.html}{reStructuredText}
documentation for details.

\sphinxstepscope


\chapter{Teza}
\label{\detokenize{rozdzial1:teza}}\label{\detokenize{rozdzial1::doc}}
\sphinxAtStartPar
W dowolnym trójkącie prostokątnym, suma pól kwadratów zbudowanych na
przyprostokątnych trójkąta prostokątnego równa jest polu kwadratu
zbudowanego na przeciwprostokątnej tego trójkąta.

\sphinxAtStartPar
lub

\sphinxAtStartPar
W trójkącie prostokątnym suma kwadratów długości przyprostokątnych jest
równa kwadratowi długości przeciwprostokątnej tego trójkąta.


\begin{savenotes}\sphinxattablestart
\sphinxthistablewithglobalstyle
\centering
\begin{tabulary}{\linewidth}[t]{TTTT}
\sphinxtoprule
\sphinxstyletheadfamily 
\sphinxAtStartPar
\sphinxstylestrong{Kategoria}
&\sphinxstyletheadfamily 
\sphinxAtStartPar
\sphinxstylestrong{Opis}
&\sphinxstyletheadfamily 
\sphinxAtStartPar
\sphinxstylestrong{Wzór lub
Kluczowa Idea}
&\sphinxstyletheadfamily 
\sphinxAtStartPar
\sphinxstylestrong{Kontekst /
Postacie}
\\
\sphinxmidrule
\sphinxtableatstartofbodyhook
\sphinxAtStartPar
\sphinxstylestrong{Teza
(T
wierdzenie)}
&
\sphinxAtStartPar
W trójkącie
prostokątnym
suma kwadratów
pr
zyprostokątnych
(\$a, b\$) jest
równa
kwadratowi
prze
ciwprostokątnej
(\$c\$).
&
\sphinxAtStartPar
\$a\textasciicircum{}2 + b\textasciicircum{}2 =
c\textasciicircum{}2\$
&
\sphinxAtStartPar
Pitagoras (VI
w. p.n.e.),
choć znane
wcześniej
(Egipt,
Babilonia).
\\
\sphinxhline
\sphinxAtStartPar
\sphinxstylestrong{In
terpretacja}
&
\sphinxAtStartPar
Suma pól
kwadratów
zbudowanych na
pr
zyprostokątnych
jest równa polu
kwadratu
zbudowanego na
przec
iwprostokątnej.
&
\sphinxAtStartPar
Pole(\$K\_a\$) +
Pole(\$K\_b\$) =
Pole(\$K\_c\$)
&
\sphinxAtStartPar
Wizualizacja
geometryczna
(jak na
Rysunku 1 w
tekście).
\\
\sphinxhline
\sphinxAtStartPar
\sphinxstylestrong{Dowody}
&
\sphinxAtStartPar
Istnieje bardzo
wiele (ponad
350) dowodów
twierdzenia, o
różnym
charakterze
(algebraiczne,
geometryczne).
&
\sphinxAtStartPar
Dowód
„układanka”
(kwadrat o boku
\$a+b\$), dowód
geometryczny
Euklidesa
(oparty na
polach).
&
\sphinxAtStartPar
Euklides
(\sphinxstyleemphasis{Elementy}),
Szczepan
Jeleński.
\\
\sphinxhline
\sphinxAtStartPar
\sphinxstylestrong{Twierdzenie
Odwrotne}
&
\sphinxAtStartPar
Jeśli boki
trójkąta (a, b,
c) spełniają
warunek \$a\textasciicircum{}2 +
b\textasciicircum{}2 = c\textasciicircum{}2\$, to
trójkąt ten
jest
prostokątny.
&
\sphinxAtStartPar
Używane
praktycznie do
wyznaczania
kąta prostego
(np. trójkąt
3\sphinxhyphen{}4\sphinxhyphen{}5).
&
\sphinxAtStartPar
Starożytne
Chiny, Indie,
Egipt
(zastosowania
praktyczne).
\\
\sphinxbottomrule
\end{tabulary}
\sphinxtableafterendhook\par
\sphinxattableend\end{savenotes}

\sphinxstepscope


\chapter{2. Interpretacja}
\label{\detokenize{rozdzial2:interpretacja}}\label{\detokenize{rozdzial2::doc}}
\sphinxAtStartPar
\sphinxincludegraphics[width=2.08333in,height=2.3125in]{{image1}.png}

\sphinxAtStartPar
Rysunek 1. Interpretacja twierdzenia Pitagorasa

\sphinxAtStartPar
Oto interpretacja geometryczna: jeżeli na bokach trójkąta prostokątnego
zbudujemy kwadraty, to suma pól kwadratów zbudowanych na
przyprostokątnych tego trójkąta jest równa polu kwadratu zbudowanego na
przeciwprostokątnej. W sytuacji na rysunku obok: suma pól kwadratów
„fioletowego” i „zielonego” jest równa polu kwadratu „czerwonego”.

\sphinxstepscope


\chapter{3. Dowody}
\label{\detokenize{rozdzial3:dowody}}\label{\detokenize{rozdzial3::doc}}
\sphinxAtStartPar
Liczba różnych dowodów twierdzenia Pitagorasa jest przytłaczająca,
według niektórych źródeł przekracza 350. Euklides w Elementach podaje
ich osiem, kolejne pojawiały się na przestrzeni wieków i pojawiają aż po
dni dzisiejsze.

\sphinxAtStartPar
Niektóre z dowodów są czysto algebraiczne (jak dowód z podobieństwa
trójkątów), inne mają formę układanek geometrycznych (prawdopodobny
dowód Pitagorasa), jeszcze inne oparte są o równości pól pewnych figur.
Zaprezentujemy tu jedynie kilka wybranych dowodów, do innych podajemy
odsyłacze na końcu artykułu.


\section{3.1 Dowód układanka}
\label{\detokenize{rozdzial3:dowod-ukladanka}}
\sphinxAtStartPar
Dany jest trójkąt prostokątny o bokach długości a,b i c jak rysunku z
lewej. Konstruujemy kwadrat o boku długości a + b w sposób ukazany na
rysunku z lewej, a następnie z prawej. Z jednej strony pole kwadratu
równe jest sumie pól czterech trójkątów prostokątnych i kwadratu
zbudowanego na ich przeciwprostokątnych, z drugiej zaś równe jest ono
sumie pól tych samych czterech trójkątów i dwóch mniejszych kwadratów
zbudowanych na ich przyprostokątnych. Stąd wniosek, że pole kwadratu
zbudowanego na przeciwprostokątnej jest równe sumie pól kwadratów
zbudowanych na przyprostokątnych.

\sphinxAtStartPar
Szczepan Jeleński w książce Śladami Pitagorasa przypuszcza, że w ten
sposób mógł udowodnić swoje twierdzenie sam Pitagoras.

\sphinxAtStartPar
Powyższy dowód, choć prosty, nie jest elementarny w tym sensie, że jego
poprawność wymaga uprzedniego uzasadnienia, że pole kwadratu złożonego z
trójkątów i mniejszych kwadratów jest równe sumie pól tych figur. Może
się to wydawać oczywiste, jednak dowód tego faktu wymaga uprzedniego
zdefiniowania pola, na przykład poprzez konstrukcję miary Jordana.


\section{3.2 Dowód czysto geometryczny}
\label{\detokenize{rozdzial3:dowod-czysto-geometryczny}}
\sphinxAtStartPar
Następujący dowód znajduje się w Elementach Euklidesa i oparty jest na
spostrzeżeniu, że pola dwu mniejszych kwadratów zbudowanych na
przyprostokątnych trójkąta prostokątnego ΔABC są równe polom
odpowiednich prostokątów na jakie wysokość CD dzieli kwadrat zbudowany
na przeciwprostokątnej.

\sphinxAtStartPar
Dla dowodu zauważmy, że pole kwadratu ACJK jest równe podwojonemu polu
trójkąta ΔKAB \textendash{} podstawą trójkąta ΔKAB jest bok KA kwadratu, a wysokość
trójkąta jest równa bokowi CA tego kwadratu. Podobnie, pole prostokąta
AEGD jest równe podwojonemu polu trójkąta ΔCAE \textendash{} podstawą trójkąta ΔCAE
jest bok AE prostokąta, a wysokość trójkąta jest równa bokowi EG
prostokąta. Jednak trójkąty ΔKAB i ΔCAE są przystające, co wynika z
cechy „bok\sphinxhyphen{}kąt\sphinxhyphen{}bok” \textendash{} | KA | = | CA | , | AB | = | AE | i kąt
jest równy kątowi \textendash{} a zatem mają równe pola, skąd wynika, że pole
kwadratu ACJK jest równe polu prostokąta AEGD.

\sphinxAtStartPar
Analogicznie (rozważając trójkąty ΔCBF i ΔHBA można udowodnić, że pole
kwadratu CBHI jest równe polu prostokąta BFGD. Stąd, suma pól obu
kwadratów równa jest polu kwadratu AEFB.

\sphinxstepscope


\chapter{4. Uwagi}
\label{\detokenize{rozdzial4:uwagi}}\label{\detokenize{rozdzial4::doc}}
\sphinxAtStartPar
Następujący dowód znajduje się w Elementach Euklidesa i oparty jest na
spostrzeżeniu, że pola dwu mniejszych kwadratów zbudowanych na
przyprostokątnych trójkąta prostokątnego ΔABC są równe polom
odpowiednich prostokątów na jakie wysokość CD dzieli kwadrat zbudowany
na przeciwprostokątnej.

\sphinxAtStartPar
Dla dowodu zauważmy, że pole kwadratu ACJK jest równe podwojonemu polu
trójkąta ΔKAB \textendash{} podstawą trójkąta ΔKAB jest bok KA kwadratu, a wysokość
trójkąta jest równa bokowi CA tego kwadratu. Podobnie, pole prostokąta
AEGD jest równe podwojonemu polu trójkąta ΔCAE \textendash{} podstawą trójkąta ΔCAE
jest bok AE prostokąta, a wysokość trójkąta jest równa bokowi EG
prostokąta. Jednak trójkąty ΔKAB i ΔCAE są przystające, co wynika z
cechy „bok\sphinxhyphen{}kąt\sphinxhyphen{}bok” \textendash{} | KA | = | CA | , | AB | = | AE | i kąt
jest równy kątowi \textendash{} a zatem mają równe pola, skąd wynika, że pole
kwadratu ACJK jest równe polu prostokąta AEGD.

\sphinxAtStartPar
Analogicznie (rozważając trójkąty ΔCBF i ΔHBA można udowodnić, że pole
kwadratu CBHI jest równe polu prostokąta BFGD. Stąd, suma pól obu
kwadratów równa jest polu kwadratu AEFB.



\renewcommand{\indexname}{Indeks}
\printindex
\end{document}